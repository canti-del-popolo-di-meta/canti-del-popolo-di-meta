\documentclass[11pt]{book}
\usepackage[utf8]{inputenc}
\usepackage[italian]{babel}
\usepackage{verse}

% http://tex.stackexchange.com/questions/153262/latex-poetry-anthology-templates
% http://samwhited.github.com/poetrytex/
\usepackage[numberpoems, clearpageafterpoem, useincipits]{poetrytex}

% use the PA5 paper size
\usepackage[paperwidth=140mm,paperheight=210mm]{geometry}

\newenvironment{bottompar}{\par\vspace*{\fill}}{\clearpage}

\renewcommand{\pttitle}{Canti del popolo di Meta}
\renewcommand{\ptsubtitle}{}
\renewcommand{\ptauthor}{Roberto Reale}
\renewcommand{\ptdate}{\today}
\renewcommand{\ptdedication}{A Laura Celentano\\*
di Meta\\*
per la quale imparammo\\*
quanto di greca umanità ci fosse in noi\\*
quanto di disumano in Grecia}

\begin{document}

\maketitle
\makededication

% number pages with small roman numerals (i, ii, iii, iv...)
\frontmatter

\newpage
\thispagestyle{empty}

\begin{bottompar}
This work is licensed under a \emph{Creative Commons Attribution-ShareAlike 4.0 International License}.
\end{bottompar}

\section{Nota al lettore}

TODO

\newpage
\thispagestyle{empty}

% TOP
\renewcommand*{\topname}{Canzune} % Name for the table of poems
\maketop

% start numbering pages with normal arabic numerals.
\mainmatter

% 1
\begin{poem}{’A bella mia se chiamma}{}
\settowidth{\versewidth}{Il tuo cuore dallo a chi glie l’hai promesso}
\begin{altverse}
La bella mia si chiama Maddalena;\\*
un anno intero m’ha fatto penare.\\*
Con una mano mi fa cenno e dice: — Vieni!\\*
e poi con l’altra mi tiene lontano.\\*
Basta con i capricci: cos’hai in testa?\\*
Il tuo cuore dallo a chi glie l’hai promesso.\\*
E se l’hai promesso a me, tienici fede;\\*
se l’hai promesso a un altro, non mancare.
\end{altverse}
\end{poem}

\begin{poem}{Aggio saputo ca la morta vene}{}
\settowidth{\versewidth}{Tu che sei bella non puoi star tranquilla}
\begin{altverse}
— M’hanno detto che viene la morte\\*
per pigliarsi le donne più belle.\\*
Tu che sei bella non puoi star tranquilla:\\*
queste bellezze a chi le vuoi lasciare?\\*
Lasciale, deh, a chi ti vuole bene:\\*
e io verbigrazia non ti voglio male.\\*
— Le lascerei più tosto alla negra terra\\*
per non lasciarle a te, cuore di cane.
\end{altverse}
\end{poem}

% 3
\begin{poem}{Aggio saputo ca si’ fatta santa}{}
\settowidth{\versewidth}{Così ti possano ingannare i tuoi parenti}
\begin{altverse}
Ho saputo che ti sei fatta santa;\\*
pure alla chiesa vai, assiduamente.\\*
La coroncina, neh, che porti avanti,\\*
tu ce la porti a inganno della gente.\\*
A me tu m’hai ingannato, povero amante:\\*
così ti possano ingannare i tuoi parenti.\\*
Ti dico questo, a te: d’ora in avanti\\*
troppa folla ci trovi a quel convento.
\end{altverse}
\end{poem}

% 4
\begin{poem}{Aizo l’ uocchie ’ncielo}{}
\settowidth{\versewidth}{Alzo gli occhi al cielo e vedo una stella}
\begin{altverse}
Alzo gli occhi al cielo e vedo una stella:\\*
potenza d’amore — quanto sei bella!\\*
Chiamo Rosa e risponde Angiolella.\\*
— Dimmi, figliola, qual ti pare bella?\\*
Da quando le vado dietro, lei\\*
mi ci vuol far morire, infame e bella.\\*
Colei si può chiamar la vera stella:\\*
colei che porta in petto due roselle.
\end{altverse}
\end{poem}

% 5
\begin{poem}{A l’ acqu’ a l’ acqua de li ffuntanelle}{}
\settowidth{\versewidth}{All’acqua, all’acqua delle fontanelle}
\begin{altverse}
All’acqua, all’acqua delle fontanelle\\*
dove ci vanno le donne a lavare\\*
mi voglio scegliere la meglio\\*
e sempre con me la voglio portare.\\*
Tutti mi diranno: — Quanto è bella!\\*
Dove l’hai fatta ’sta caccia reale?\\*
— L’ho fatta alle porte dell’Avella\\*
dove la neve non si consuma mai.
\end{altverse}
\end{poem}

% 6
\begin{poem}{Amava ’na donna}{}
\settowidth{\versewidth}{Amavo una donna: era il suo nome Agnese}
\begin{altverse}
Amavo una donna: era il suo nome Agnese.\\*
Ne amavo un’altra: le dicevan Rosa.\\*
Un’altra ancora: si chiamava Teresa.\\*
Ma... solo Teresa mi dava qualcosa.\\*
E un giorno corsi di carriera tesa\\*
per veder quest’uccello ove si posa.\\*
E volli darle un bacio, alla Teresa:\\*
non mi saltano addosso Agnese e Rosa?
\end{altverse}
\end{poem}

% 7
\begin{poem}{’A primma vota che me cunfessaie}{}
\settowidth{\versewidth}{Io gli risposi: — Padre, e che ne sai?}
\begin{altverse}
La prima volta che mi confessai\\*
mi confessai da un santo predicatore.\\*
La prima cosa che mi domandò,\\*
mi disse: — Figliolo, fai all’amore?\\*
Io gli risposi: — Padre, e che ne sai?\\*
— Io conosco le tue intenzioni.\\*
Una penitenza te la voglio dare:\\*
va’ da lei spesso e portale più amore.
\end{altverse}
\end{poem}

% 8
\begin{poem}{Arevo piccerillo te piantaie}{}
\settowidth{\versewidth}{Tanto la bella mia ha cambiato amore}
\begin{altverse}
Albero pargoletto ti piantai\\*
ti innaffiai con il mio sudore.\\*
Il vento viene e ne strappa un ramo\\*
la fronda verde mutò di colore.\\*
Il frutto dolce mi si fece amaro:\\*
dov’è perduto quel dolce sapore?\\*
Vieni qui, morte, trovaci riparo,\\*
tanto la bella mia ha cambiato amore.
\end{altverse}
\end{poem}

% 9
\begin{poem}{Bella che stat’ a ’st’ àsteco schianato}{}
\settowidth{\versewidth}{Quando alla finestrella vi affacciate}
\begin{altverse}
Da quest’altana, bella, vi mostrate:\\*
i biondi capellucci vi sciogliete.\\*
Poi a una conchetta d’acqua li lavate\\*
e biondi più che l’oro ve li fate.\\*
Quando ve li avvolgete alla spadina\\*
più bianca e vermiglietta diventate.\\*
Poi voltate le spallucce al sole\\*
e del sole i raggi catturate.\\*
Quando alla finestrella vi affacciate\\*
me ne tirate con la calamita.
\end{altverse}
\end{poem}

% 10
\begin{poem}{Bella figliola, nun t’ arde lu core}{}
\settowidth{\versewidth}{Diglielo tu che dentro non puoi stare}
\begin{altverse}
Bella figliola, non ti arde il cuore\\*
quando la sera mi senti passare?\\*
Alluma la candela... vieni fuori;\\*
diglielo a mamma tua che vuoi filare.\\*
Se mamma tua ti dice una parola\\*
diglielo tu che dentro non puoi stare:\\*
— Là ci sta il nino mio, e io\\*
tutta di fuoco mi sento incendiare.
\end{altverse}
\end{poem}

% 11
\begin{poem}{Càrcere fatt’ a làmmi’ e nun a trave}{}
\settowidth{\versewidth}{Assedia la porta di spranghe e di chiodi}
\begin{altverse}
Carcere fatto a volta e non a trave,\\*
io meschinello dentro mi ci trovo.\\*
Carcere che spezza gli uomini valenti:\\*
ci entrano e ci lasciano la tempra.\\*
Viene il carceriere con le chiavi,\\*
assedia la porta di spranghe e di chiodi.\\*
Misero me, se carcerato vado;\\*
chiamo amici e parenti e non ne trovo.
\end{altverse}
\end{poem}

% 12
\begin{poem}{Ch’ addore de garuòfene che sento}{}
\settowidth{\versewidth}{Ce ne sta un mazzo in petto a nenna mia}
\begin{altverse}
Che odore di garofani che sento...!\\*
Ce ne sta un mazzo in petto a nenna mia!\\*
Questi non son garofani, né altro:\\*
è respiro di nenna mia che odora tanto!\\*
Se qualcuno me la guarda con la voglia\\*
glie lo faccio pigliare l’olio santo!\\*
E se mamma tua di noi non è contenta\\*
tu ti fai monachella e io monaco santo.
\end{altverse}
\end{poem}

% 13
\begin{poem}{Chi t’ ha ditto ca n’ aggio ’nnammurate}{}
\settowidth{\versewidth}{Chi è venuto a dirti che non tengo innamorate}
\begin{altverse}
Chi è venuto a dirti che non tengo innamorate?\\*
Sette ne tengo, pronte al mio comando.\\*
Una la tengo a Massa e un’altra a Capri\\*
un’altra alla piana di Vico\\*
un’altra ancora sopra Massaquano\\*
un’altra sta sopra Santo Vito\\*
e un’altra me la tengo qui vicino:\\*
a lei confido le pene mie d’amore.
\end{altverse}
\end{poem}

% 14
\begin{poem}{Dummèneca matina de li Pparme}{}
\settowidth{\versewidth}{Stare con nenna mia la notte e il giorno}
\begin{altverse}
Domenica mattina delle Palme\\*
non si fosse mai per me levato giorno!\\*
Alla chiesa andai, per portar la palma:\\*
mi vidi almeno cento sbirri attorno.\\*
Gli dissi: — Non ho fatto danno,\\*
non ho ucciso nessuno a ’ste contrade!\\*
Da Napoli è venuta la condanna:\\*
stare con nenna mia la notte e il giorno.
\end{altverse}
\end{poem}

% 15
\begin{poem}{Faccia de nucepièrzeco sciuruto}{}
\settowidth{\versewidth}{Come se non ci fossero più al mondo innamorate}
\begin{altverse}
Faccia di noce persica sfiorita:\\*
pare che la gelata ti ha seccato!\\*
Mi vieni dietro come un disperato\\*
come se non ci fossero più al mondo innamorate.\\*
Racconti in giro che non m’hai voluta:\\*
perché non dici che io ti ho lasciato?
\end{altverse}
\end{poem}

% 16
\begin{poem}{Fenestelluzza tutta rentagliata}{}
\settowidth{\versewidth}{La notte stai aperta e il giorno serrata}
\begin{altverse}
Finestrella mia, intarsiata tutta\\*
come fronda di gelsomino,\\*
la notte stai aperta e il giorno serrata:\\*
lo fai apposta per tormento mio.\\*
Finestra, se ti avessi in altro luogo\\*
in centomila schegge ti farei.\\*
Finestra, che tu possa ardere nel fuoco\\*
perché ci tieni chiusa a nenna mia.
\end{altverse}
\end{poem}

% 17
\begin{poem}{Figliola, ca si’ bell’ e nun abballe}{}
\settowidth{\versewidth}{D’uopo è una botta del mio gran martello}
\begin{altverse}
Figliola, che sei bella ma non balli:\\*
attenta, ché hai uno sgarro nella gonna!\\*
E se cerchi sotto, troverai una falla:\\*
per turarla trenta rotoli di pelle,\\*
di stoppa ne consumi dieci balle\\*
e di pece non ti basta un bastimento.\\*
Ma se tu la vuoi sanare questa falla\\*
d’uopo è una botta del mio gran martello!
\end{altverse}
\end{poem}

% 18
\begin{poem}{Figliola, dàtte vot’ a ’sta ienesta}{}
\settowidth{\versewidth}{Figliola, e appènditici a ’sta ginestra}
\begin{altverse}
Figliola, e appènditici a ’sta ginestra!\\*
Al tempo che t’amavo io ero pazzo.\\*
Tua madre parla d’imbastire parentela:\\*
tu non sei buona neppure a far da serva.\\*
Se non sai di creanza io te ne insegno:\\*
meglio che parli poco o tu ne buschi!\\*
Non prende un calesse tante botte\\*
quanti cefali ci vanno nella nassa.
\end{altverse}
\end{poem}

% 19
\begin{poem}{Figliola, te petriene ca si’ bella}{}
\settowidth{\versewidth}{Più storta d’una storpia e ancora parli}
\begin{altverse}
Figliola, ti lusinghi d’esser bella...\\*
che in beltà nessuna ti è compagna.\\*
A berlina io ti terrei su un rotella\\*
tutta piena di polvere e di palle.\\*
La bocca tieni della cinciarella\\*
il naso uguale uguale a un pappagallo.\\*
Più còncava pure d’una vecchierella\\*
più storta d’una storpia e ancora parli!
\end{altverse}
\end{poem}

% 20
\begin{poem}{Figliulo cu’ ’sta capa ’ncarnatella}{}
\settowidth{\versewidth}{Ne hai fatte di carambole e di imbrogli}
\begin{altverse}
Figliolo, la tua faccia rubizza\\*
per queste case ci sale e ci scende.\\*
Mandi imbasciata a mille fanciulle:\\*
nessuna mamma ti vuol dar la figlia.\\*
Ne hai fatte di carambole e di imbrogli:\\*
non c’è incappata una nel tuo vischio!\\*
Questo ti dico a te, figliolo bello:\\*
vecchio ti fai e nessuna ti piglia.
\end{altverse}
\end{poem}

% 21
\begin{poem}{Galera cu’ ’sti nuove peramiente}{}
\settowidth{\versewidth}{— Non ho paura d’acqua né di vento}
\begin{altverse}
Galera dai nuovi paramenti,\\*
stai scoperta cento miglia a mare.\\*
Vele d’oro e remi d’argento\\*
quanto più porti tu più bella pari!\\*
Galera, se m’incappi sotto vento\\*
giuro che le vele ti faccio calare.\\*
— Non ho paura d’acqua né di vento:\\*
tengo a nennillo mio marinaio.
\end{altverse}
\end{poem}

% 22
\begin{poem}{Ge volimm’ amar’ e la gente nu’ bonno}{}
\settowidth{\versewidth}{Vogliono che dei miei panni io me ne spogli}
\begin{altverse}
Ci vogliamo amare e la gente non vuole.\\*
Ma che ne vuole ’sta gente da me?\\*
Vogliono che dei miei panni io me ne spogli,\\*
ché non posso fare l’amore in eterno...\\*
Ma io, venisse pure la maestà di Spagna\\*
e mi dicesse: — Amore, non amare donne,\\*
e con la spada mi passasse parte a parte...\\*
Io non rinnego nenna mia per altre donne.
\end{altverse}
\end{poem}

% 23
\begin{poem}{I’ g’ era peccerillo de dui’ anne}{}
\settowidth{\versewidth}{Non ci avrei perso il bene del mio sonno}
\begin{altverse}
Io ero un fantolino di due anni\\*
e mamma me lo disse: — Non amare donne.\\*
Fino ai vent’anni la volli ubbidire\\*
e poi ai ventuno volli amar due donne.\\*
E la prima che amai era una tiranna,\\*
la seconda mi ci fece perdere il sonno.\\*
Avessi dato retta alla mia mamma!\\*
Non ci avrei perso il bene del mio sonno.
\end{altverse}
\end{poem}

% 24
\begin{poem}{Iett’ a la chianca pe’ m’ abbuscà’ ’n uosso}{}
\settowidth{\versewidth}{Non piango che son caduto dentro al fosso}
\begin{altverse}
Andai al macello a procurarmi un osso,\\*
che ne potessi far minestra grassa.\\*
Ci stava un cane — giuro — assai grosso:\\*
mi salta addosso come un satanasso.\\*
Vado più avanti, ci incontro un fosso:\\*
io spingo il piede e mi trovai da basso.\\*
Non piango che son caduto dentro al fosso:\\*
piango la nenna mia che mo’ mi lascia.
\end{altverse}
\end{poem}

% 25
\begin{poem}{Iett’ a piscar’ a nu pìcculu mare}{}
\settowidth{\versewidth}{Io voglio abbandonarlo il mio mestiere}
\begin{altverse}
Andai a pescare al più piccolo mare\\*
credendo ch’ero il solo pescatore.\\*
Là ci trovai mille e mille marinai\\*
che non bastava il mare maggiore.\\*
Io voglio abbandonarlo il mio mestiere,\\*
non voglio mangiar pesce con sudore.\\*
Non voglio che si dica, oggi o domani:\\*
«passero vecchio in gabbia caduto».
\end{altverse}
\end{poem}

% 26
\begin{poem}{Iett’ a Roma pe me fare Papa}{}
\settowidth{\versewidth}{A mettere il piede dove non ci cape}
\begin{altverse}
A Roma andai per diventare Papa\\*
ma per la via mi ritrovai papùto.\\*
Vado per dar morsi a una senàpe:\\*
in bocca mi sento una cima di ruta.\\*
A mettere il piede dove non ci cape\\*
tu ci puoi scivolare e ti dirupi.\\*
Se vuoi sapere dove stan le capre:\\*
dove vedi aggirarsi spesso il lupo.
\end{altverse}
\end{poem}

% 27
\begin{poem}{I’ te salut’ a te, bianco palazzo}{}
\settowidth{\versewidth}{Voglia il cielo t’avessi in queste braccia}
\begin{altverse}
Io ti saluto a te, bianco palazzo;\\*
l’artefice t’ha dato tant’altezza!\\*
Io vi saluto a voi, cuscini e materassi\\*
ove riposa la vostra bellezza.\\*
Del tuo corpetto ne vorrei un nastro,\\*
solo un capello delle bionde trecce!\\*
Voglia il cielo t’avessi in queste braccia:\\*
me ne vorrei morir di contentezza.
\end{altverse}
\end{poem}

% 28
\begin{poem}{Iunne capille miei, iunne capille}{}
\settowidth{\versewidth}{Fate morir gli amanti a mille a mille}
\begin{altverse}
Biondi capelli miei, biondi capelli,\\*
solo se camminate siete bella.\\*
Fate morir gli amanti a mille a mille\\*
e poi li sanate con quest’occhi belli.\\*
In questo palazzo ce ne stanno mille;\\*
solo nennella mia è la più bella.
\end{altverse}
\end{poem}

% 29
\begin{poem}{La bella mia sta malat’ a letto}{}
\settowidth{\versewidth}{Io un’altra amante tengo e tu ci schiatti}
\begin{altverse}
La bella mia sta malata a letto:\\*
tiene la mala pasqua che la batte.\\*
S’è mangiate due galline cotte\\*
s’è rosicchiate l’ossa come un gatto\\*
s’è mangiati due sacchi di confetti\\*
s’è prosciugate due sorgenti d’acqua.\\*
Ti lusingavi di farmi dispetto:\\*
io un’altra amante tengo e tu ci schiatti.
\end{altverse}
\end{poem}

% 30
\begin{poem}{La tòrtora si perde la cumpagna}{}
\settowidth{\versewidth}{Un giorno intero se ne sta intristita}
\begin{altverse}
La tortora se perde la compagna\\*
un giorno intero se ne sta intristita.\\*
Poi trova l'acqua fresca, vi si bagna,\\*
e se la beve tutta annuvolata.\\*
Poi se ne va alla cima di un monte\\*
e là si piange le disgrazie sue.\\*
Così è la nenna quando si scompagna:\\*
smarrisce ogni gusto e ogni piacere.
\end{altverse}
\end{poem}

% 31
\begin{poem}{Lu spasso de lu ventre}{}
\settowidth{\versewidth}{Il piacere dei maestri: gli scolari}
\begin{altverse}
Il piacere del ventre: il pane.\\*
Il piacere del vino: la gola.\\*
Il piacere della lepre: la tana.\\*
Il piacere delle serpi: i cepponi.\\*
Il piacere dei maestri: gli scolari.\\*
Il piacere dei libri: i dottori.\\*
Sei vuoi sapere il miglior piacere:\\*
quando mi prendo piacere con voi.
\end{altverse}
\end{poem}

% 32
\begin{poem}{Mamma, che de fìgliet’ haie paura}{}
\settowidth{\versewidth}{Io scasso le porte e la vengo a trovare.}
\begin{altverse}
Mamma, che di tua figlia hai paura,\\*
voglio vedere se la sai guardare.\\*
Disloca macigni, fabbrica le mura,\\*
davanti alla porta ergile una torre.\\*
Serrala poi con sette serrature\\*
come una cassa piena di tesori.\\*
Ma se qualche giorno mi tenta la sorte\\*
io scasso le porte e la vengo a trovare.
\end{altverse}
\end{poem}

% 33
\begin{poem}{Mamm’, ’u scurpione m’ ha muzzecato}{}
\settowidth{\versewidth}{Madre, senza spargere sangue mi ha ferita}
\begin{altverse}
Madre, lo scorpione mi ha morsicata.\\*
Tu non sai la ferita che mi fece.\\*
Tu non sai il dolore che mi diede.\\*
Madre, senza spargere sangue mi ha ferita.\\*
Dall’inciarmatore sì, ci sono andata.\\*
Medicine? Nessuna ci ha potuto.\\*
Sentite la risposta che mi ha dato:\\*
fatti sanare da chi ti ci ha ferita.
\end{altverse}
\end{poem}

% 34
\begin{poem}{Maretatella mia, maretatella}{}
\settowidth{\versewidth}{Se t’hai da maritare, fatti il corredo}
\begin{altverse}
Maritatella mia, maritatella,\\*
chi te l’ha datto quel brutto marito?\\*
E tuo marito è brutto e tu sei bella:\\*
su, vattelo a cambiare giù al canneto.\\*
Nel canneto ci stanno le canne,\\*
sulle galere gli alberi e le antenne:\\*
se t’hai da maritare, fatti il corredo,\\*
ché l’uccello non vola senza penne.*
\end{altverse}
\end{poem}

% 35
\begin{poem}{Me voglio fare ’na scuppett’ a miccia}{}
\settowidth{\versewidth}{Ci voglio andare quattro giorni a caccia}
\begin{altverse}
Mi voglio fare uno schioppétto a miccia:\\*
ci voglio andare quattro giorni a caccia.\\*
Sotto a una ripa ci trovai un riccio:\\*
quattro botte gli meno e non lo faccio.\\*
E subito subito ci cambio la miccia,\\*
ma lo schioppétto mi fa catenaccio.\\*
Non piango ché ho perduto il riccio:\\*
piango a nennella mia che non s’affaccia.
\end{altverse}
\end{poem}

% 36
\begin{poem}{Mìsero me! so’ pòvero marvizzo}{}
\settowidth{\versewidth}{Così è la donna quando s’incapriccia}
\begin{altverse}
Misero me! Son povero tordo,\\*
tengo la testa una vampa di fuoco.\\*
Viene il cacciatore e mi stordisce:\\*
di questa vita mia fa festa e gioco.\\*
Poi m’ha per le mani e mi finisce:\\*
mi ci mette in quell’ardente fuoco.\\*
Così è la donna quando s’incapriccia:\\*
mi porta all’orto a cogliere viole!
\end{altverse}
\end{poem}

% 37
\begin{poem}{Mìsero me! so’ povero surdato}{}
\settowidth{\versewidth}{Vo’ far pace con lei che m’ha tradito}
\begin{altverse}
Misero me! Son povero soldato;\\*
la mia libertà io l'ho perduta.\\*
Ho mangiato pane di forzato\\*
e acqua con i vermi ne ho bevuta.\\*
Mamma e padre m’hanno abbandonato;\\*
i meglio amici miei m’hanno tradito.\\*
Se il cielo me ne scampa da soldato\\*
vo’ far pace con lei che m’ha tradito.
\end{altverse}
\end{poem}

% 38
\begin{poem}{Musso d’ argiento mio}{}
\settowidth{\versewidth}{Voglio più bene a voi che a un altro amante}
\begin{altverse}
Muso d’argento mio, musetto d’argento,\\*
sei una bambina e già tieni due amanti.\\*
Il primo è d’oro e il secondo d’argento.\\*
Ma tu dimmelo, nenna: a chi vuoi bene?\\*
— Io a quello d’oro non lo stimo niente,\\*
quello d’argento solo è il vero amante.\\*
E se per sorte si cambieranno i venti\\*
voglio più bene a voi che a un altro amante.
\end{altverse}
\end{poem}

% 39
\begin{poem}{’Na nenna bella quanno se marita}{}
\settowidth{\versewidth}{Si pregia d’aver tolto un gran tesoro}
\begin{altverse}
La bella nenna quando si marita\\*
si pregia d’aver tolto un gran tesoro.\\*
Ma passati presto gli sponsali\\*
eccotela chioccia spennacchiata!\\*
Poi si rigira in faccia al marito:\\*
che non mi fossi giammai maritata!\\*
Meglio assai restare vecchia zita:\\*
non patirei le pene che mi date.
\end{altverse}
\end{poem}

% 40
\begin{poem}{Nàpule bell’ e Surriento cevile}{}
\settowidth{\versewidth}{Me la porto a Sorrento a rifiorire}
\begin{altverse}
Napoli bella e Sorrento civile;\\*
Sorrento più mi giova di vantare.\\*
A Sorrento c’è l’aria gentile;\\*
chi sta malato là si può sanare.\\*
Se sta malata tanto la mia bella\\*
me la porto a Sorrento a rifiorire.
\end{altverse}
\end{poem}

% 41
\begin{poem}{Nàpule bello ne vurria}{}
\settowidth{\versewidth}{Di ’sta boccuccia ne vorrei un bacio}
\begin{altverse}
Napoli bella ne vorrei le case\\*
e di Salerno il farro ed il riso.\\*
Di Castellammare le tele pregiate\\*
e una fanciulla a farne camicie.\\*
D’Arola ne vorrei le cerase\\*
e di Sorrento i fichi paradiso.\\*
Di ’sta boccuccia ne vorrei un bacio:\\*
muoio contento e vado in paradiso!
\end{altverse}
\end{poem}

% 42
\begin{poem}{Nenna che stai’ affacciat’ a ’sta fenesta}{}
\settowidth{\versewidth}{Sia maledetto chi ci vuol dormire}
\begin{altverse}
Nenna, che t’affacci\\*
da questa finestra:\\*
fammi grazia,\\*
non nasconderti.\\*
Ma sciogliti\\*
un capello dalle trecce.\\*
Lascialo scendere\\*
ché voglio salire.\\*
E quando sarò arrivato\\*
alla tua finestra\\*
prendimi in braccio\\*
e portami a dormire.\\*
Però quando saremo nel tuo letto\\*
sia maledetto chi ci vuol dormire!
\end{altverse}
\end{poem}

% 43
\begin{poem}{Nu iuorno me chiammaie ’na massesa}{}
\settowidth{\versewidth}{Ma sapeste alla fine la Massese: nel piatto}
\begin{altverse}
E mi chiamò un giorno una Massese,\\*
che volassi in un salto alla sua casa!\\*
Io mi lanciai di carriera tesa\\*
e trovo lei, la porta aperta, e: — Trase.\\*
E poi mi accoglie col desco imbandito:\\*
un coltelluccio, il pane ed il cacio.\\*
Ma sapeste alla fine la Massese: nel piatto\\*
non vuol lasciarci proprio niente!
\end{altverse}
\end{poem}

% 44
\begin{poem}{Nu iuorno me ’mmarcaie cu’ nu turrese}{}
\settowidth{\versewidth}{Che se era donna non glie lo negavo un bacio}
\begin{altverse}
Io mi imbarcai un dì con un Torrese,\\*
credendo di arricchire la mia casa.\\*
Lui era bello e tanto nobile e cortese\\*
che se era donna non glie lo negavo un bacio.\\*
A Torre mi faceva il bel sembiante\\*
e in Sardegna poi mi fa servir da mozzo.\\*
Chi ci naviga più con i Torresi!\\*
Preferisco fare lo scaricatore al porto.\\*
E vo’ pregare Dio e santa Rosa\\*
che mi lascino tornar presto al paese:\\*
ché se l’ingaggio mi durava un altro mese\\*
manco la pelle riportavo a casa.
\end{altverse}
\end{poem}

% 45
\begin{poem}{Nu iuorno me ’mparai caudararo}{}
\settowidth{\versewidth}{Ché quanti più ne cerco, di buchi, più ne trovo}
\begin{altverse}
Un giorno mi imparai calderaio: e questa\\*
è la meglio arte che mi trovo.\\*
Ecco che me ne andavo per Lustrano:\\*
— Caldaie vecchie ve le metto a nuovo!\\*
Viene una donna col tegame in mano:\\*
— Sie’ mastro mio, conciamela buona!\\*
Io ne vorrei sgranare a più non posso,\\*
ché quanti più ne cerco, di buchi, più ne trovo.
\end{altverse}
\end{poem}

% 46
\begin{poem}{Nu iuorno stevo cu’ li mieie penziere}{}
\settowidth{\versewidth}{Scappa la lampada e se ne svelle il lume}
\begin{altverse}
Io stavo un giorno con i miei pensieri,\\*
a san Francesco, e tra me e me pregavo.\\*
Dicevo paternostri e misereri\\*
come al solito mio io soglio fare.\\*
Scappa la lampada e se ne svelle il lume:\\*
e io che stavo sotto l’ebbi da pagare.\\*
Vedete quant’è grande la fortuna mia:\\*
voglio far bene, finisco a fare male!
\end{altverse}
\end{poem}

% 47
\begin{poem}{Nun aggio comme fà’ pe’ te parlare}{}
\settowidth{\versewidth}{C’è il letto fatto, se ti vuoi coricare}
\begin{altverse}
Ma come faccio per poterti parlare?\\*
Mi voglio vestire da pio cappuccino\\*
e alla tua porta poi vengo a bussare:\\*
— Fammi la carità, figliola mia.\\*
E lei mi dice: — Non ho che dare.\\*
Mi son finiti il pane ed il vino.\\*
C’è il letto fatto, se ti vuoi coricare.\\*
— Io non cercavo altro, nenna mia!
\end{altverse}
\end{poem}

% 48
\begin{poem}{Nun ze pozza palazzo fravecare}{}
\settowidth{\versewidth}{Se manca il gran concorso della gente}
\begin{altverse}
Palazzo nuovo non si può fabbricare\\*
se non s’inizia dalle fondamenta.\\*
Chi saprebbe arare la terra\\*
se vengon meno i sette alimenti?\\*
Alla chiesa, poi, come predicare\\*
se manca il gran concorso della gente?\\*
In letto fresco non ti ci coricare:\\*
a meno che...\\*
non ti ci trovi una nenna valente!
\end{altverse}
\end{poem}

% 49
\begin{poem}{O mamma, mamma, Francischiello voglio}{}
\settowidth{\versewidth}{O non ci prenderanno forse per gemelli}
\begin{altverse}
O mamma, mamma!\\*
Io a Francischiello voglio.\\*
A Francischiello voglio\\*
e a Francischiello mi piglio.\\*
E quando siamo all’altare\\*
a dire: «Sì, voglio!»,\\*
o non ci prenderanno forse per gemelli?\\*
Lui è caruccetto e io delicatella:\\*
siamo tessuti d'un sol fil di seta.\\*
Io per dispetto me lo tengo a fianco\\*
e chi non ce lo può vedere,\\*
che lo strazi l’invidia.
\end{altverse}
\end{poem}

% 50
\begin{poem}{Palazziello de fierro ben guarnuto}{}
\settowidth{\versewidth}{Palazzo amato, di ferro, ben guarnito}
\begin{altverse}
Palazzo amato, di ferro, ben guarnito,\\*
che non ci posson nulla cannonate.\\*
Io ci vorrei fare appena uno sternuto\\*
e già il palazzotto è sgarrupato.\\*
La gente dentro va chiamando aiuto:\\*
— Chi è quel giovanotto tanto armato?\\*
Fate festa, ché ho vinto\\*
la nenna mia su seicento innamorati.
\end{altverse}
\end{poem}

% 51
\begin{poem}{Rusella, che nascist’ a lu risierto}{}
\settowidth{\versewidth}{Sto con la nera sorte, ma non son morto}
\begin{altverse}
Rosella che nascesti nel deserto,\\*
mai da me irrorata tu fosti.\\*
Non farmi andare per il mondo sperso\\*
come la nave che va cercando porto.\\*
Alla nave se perde il trinchetto\\*
altro non resta che cercarsi un porto.\\*
Agli strazi tuoi io non sto soggetto:\\*
sto con la nera sorte, ma non son morto.
\end{altverse}
\end{poem}

% 52
\begin{poem}{Se ne scenne lu sole e se fa notte}{}
\settowidth{\versewidth}{Penso ch’è fatta notte e non lo vedo}
\begin{altverse}
Inclina il sole e si fa notte;\\*
chi mi vuol bene stasera non vedo.\\*
Il sudore mi bagna della morte;\\*
penso ch’è fatta notte e non lo vedo.\\*
Alla porta lascio la risposta;\\*
finestrella senza luce di candela,\\*
io vi lascio con la buona notte:\\*
per me non mancherà la buona sera.
\end{altverse}
\end{poem}

% 53
\begin{poem}{Sera ge iette e lu marito g’ era}{}
\settowidth{\versewidth}{Disse il marito a chi gli dorme al fianco}
\begin{altverse}
Di sera ci andai, ci trovai il marito\\*
e me la feci una mala nottata!\\*
Mi nascosi per bene sotto al letto:\\*
come un ragno strisciavo per casa.\\*
Disse il marito a chi gli dorme al fianco:\\*
— Perché tanto rumore per la casa?\\*
— Dormi, marito mio, ti venga bene:\\*
questa è la gatta che sempre s’infila.
\end{altverse}
\end{poem}

% 54
\begin{poem}{Sera passaie pe’ dinto a nu vico}{}
\settowidth{\versewidth}{— Sta’ attento costì, bada a quei meloni}
\begin{altverse}
Una sera passai nel ventre di un vico:\\*
vedo quest’albero colmo di frutti.\\*
Stendo le mani, lo scuoto, ne cadono\\*
pesche più grandi di meloni!\\*
Se ne addiede il padrone da lontano:\\*
— Sta’ attento costì, bada a quei meloni.\\*
E io timoroso delle melanzane\\*
coglievo noci persiche e meloni.
\end{altverse}
\end{poem}

% 55
\begin{poem}{Sera passai’ e tu, bella, durmive}{}
\settowidth{\versewidth}{E in mezzo al giardino un albero pregiato}
\begin{altverse}
Di sera passavo e tu, bella, dormivi.\\*
Io in tutto il tuo giardino camminai.\\*
E in mezzo al giardino un albero pregiato,\\*
ma per scrupolo mio me ne scansai.\\*
Con te poi fui più cortese ancora:\\*
ti eri scoperta e io ti ricoprii.\\*
Di una cosa soltanto ci rimasi male:\\*
il fuoco era acceso e io non mi scaldai.
\end{altverse}
\end{poem}

% 56
\begin{poem}{Sia beneritto chi fece lu munno}{}
\settowidth{\versewidth}{Poi fece il mare, tutt’intorno intorno}
\begin{altverse}
Sia benedetto chi fabbricò il mondo,\\*
ché chi lo fece sì che seppe fare!\\*
Fece prima la notte e dopo il giorno,\\*
e poi li seppe crescere e scemare;\\*
poi fece il mare, tutt’intorno intorno,\\*
e fece i bastimenti a navigare.\\*
E poi fece te, colomba lucente,\\*
e ti ci fece per il mio tormento.
\end{altverse}
\end{poem}

% 57
\begin{poem}{Si t’ haie da ’nzurà’}{}
\settowidth{\versewidth}{Così se devi farle in dono una sottana}
\begin{altverse}
Sposarti devi?\\*
E allora ascolta: pigliatela bella,\\*
però non troppo. Ché poi ti fa paura.\\*
Pigliala bruna e generosa,\\*
ma che sia delicata alla cintura.\\*
Così se devi farle in dono una sottana\\*
risparmi fodera e seta e cucitura.\\*
E se vuoi farle poi un’abbracciatella:\\*
come abbracciassi un mazzo di fiori!
\end{altverse}
\end{poem}

% 58
\begin{poem}{Si vuò’ sapè’ lu spasso de li ddonne}{}
\settowidth{\versewidth}{Ma torna il marito e per bene le scuote}
\begin{altverse}
Lo vuoi saper lo spasso delle donne\\*
quando il caro marito sta in viaggio?\\*
Fuso e conocchia presto nascosti\\*
e bocconi di lusso che si fanno!\\*
S’arricciano i capelli tutt’intorno\\*
e gli vien l’uzzo di un taglio brioso!\\*
Ma torna il marito e per bene le scuote:\\*
mala cometa passa una volta l'anno.
\end{altverse}
\end{poem}

% 59
\begin{poem}{Songo venuto pe’ te parlà’ chiaro}{}
\settowidth{\versewidth}{Poi il danno fatto — è inteso — te lo pago}
\begin{altverse}
Sono venuto per parlarti chiaro.\\*
Vuoi darmela tua figlia? Che decidi?\\*
Dove la trovo me l’abbraccio e bacio;\\*
poi il danno fatto — è inteso — te lo pago.\\*
Parliamo franco, non tergiversiamo.\\*
Io coi denari ho dalla mia gli amici.\\*
Tu te ne vai con tua figlia baciata:\\*
io me ne vado contento e felice.
\end{altverse}
\end{poem}

% 60
\begin{poem}{Te gi allicuorde quanno nuie}{}
\settowidth{\versewidth}{Ti stavo dietro e ti porgevo il grano}
\begin{altverse}
Te ne ricordi quando si mieteva?\\*
Ti stavo dietro e ti porgevo il grano:\\*
facevo covoncini e te li davo,\\*
poi ti rubavo un pizzico alla mano.\\*
E lei si volta e: — Povera me!\\*
Non ne facciamo indovinar la gente.\\*
Io le risposi: — Amor non ha paura.\\*
Io so l’amore come s’ha da fare.
\end{altverse}
\end{poem}

% 61
\begin{poem}{Tenga ’na famma che me mangiarria}{}
\settowidth{\versewidth}{Cinquecent’anni con nennella bella}
\begin{altverse}
Tengo una fame che mi mangerei\\*
Napoli contornata di panelle.\\*
Tengo una sete che mi beverei\\*
Castellammare con le Fontanelle.\\*
Tengo un passo che camminerei\\*
Carotto, Pozzopiano e Trasaella.\\*
Un sonno tengo, che mi dormirei\\*
cinquecent’anni con nennella bella.
\end{altverse}
\end{poem}

% 62
\begin{poem}{Uocchie nerill’ e core de diamante}{}
\settowidth{\versewidth}{Che m’han fatto di nascosto il tradimento}
\begin{altverse}
Dolci occhi neri e cuore di diamante,\\*
come mi ti cancello dalla mente?\\*
Son quei vicini che ti tieni accanto\\*
che m’han fatto di nascosto il tradimento.\\*
Io prego a Dio e insieme a tutti i santi\\*
che mi levassero dal tuo fuoco ardente.\\*
Se gli occhi miei poi sbocciano in pianto\\*
che serve ch’io li asciughi ogni momento?
\end{altverse}
\end{poem}

% 63
\begin{poem}{Ve manno cientomìlia bone sere}{}
\settowidth{\versewidth}{Ti venisse sonno, per caso? E se ti viene}
\begin{altverse}
Vi mando centomila buone sere,\\*
e ve le mando alla napoletana!\\*
Tu scendi e metti l’olio alle candele:\\*
ché ci vogliamo spassare oggi e domani.\\*
Ti venisse sonno, per caso? E se ti viene,\\*
come ti viene così lo fai passare.\\*
Se mamma vostra poi mi volesse bene,\\*
con te mi lascerebbe... oggi e domani!
\end{altverse}
\end{poem}

% 64
\begin{poem}{Vì’ che faciette dommèneca matina}{}
\settowidth{\versewidth}{Ma dopo questo ventre mio non era sazio}
\begin{altverse}
Lo sai che ho fatto domenica mattina?\\*
A Positano me ne andai a mangiare:\\*
e mi mangiai un porcello e una gallina\\*
un agnellino senza tôr la lana\\*
di maccheroni una zuppiera sana\\*
e pure un’infornata di panelle.\\*
Ma dopo questo ventre mio non era sazio\\*
che suonava desolato come una campana.
\end{altverse}
\end{poem}

% 65
\begin{poem}{Vurrì’ addeventare ’na restina}{}
\settowidth{\versewidth}{Io addosso a te mi ci vorrei avvinghiare}
\begin{altverse}
Vorrei diventare un arbusto antico,\\*
fare una siepe impervia alla tua porta.\\*
E quando tu usciresti ogni mattina\\*
io addosso a te mi ci vorrei avvinghiare.\\*
Tu mi diresti: — Lasciami, restìna,\\*
la messa è uscita e non mi fai arrivare.\\*
— Tanto ti lascio a te, nennella mia,\\*
quando m’hai detto che mi torni a amare.
\end{altverse}
\end{poem}

% 66
\begin{poem}{Vurrì’ addeventare nu muschillo}{}
\settowidth{\versewidth}{Tu sola agli occhi miei sei la più bella.}
\begin{altverse}
Vorrei diventare un bel moschillo\\*
per librarmi in braccio alla mia bella.\\*
Me ne vorrei centellinare il sangue\\*
come una vespa l’uva moscatella!\\*
Ma quanto è buono questo miele! Cielo,\\*
non me ne strapperei fino in aeternum.\\*
In questo luogo ce ne stanno mille:\\*
tu sola agli occhi miei sei la più bella.
\end{altverse}
\end{poem}

% 67
\begin{poem}{Vurrei addeventare nu picciuòttuo}{}
\settowidth{\versewidth}{Me ne andrei in giro per ’sti palazzotti}
\begin{altverse}
Vorrei diventare un bel picciotto\\*
con un orcetto in spalla a vender acqua.\\*
Me ne andrei in giro per ’sti palazzotti:\\*
— Belle signore, chi ne vuol acqua?\\*
Si volta una figliola alla finestra:\\*
— Chi è questo piccino a vender acqua?\\*
Io le rispondo con parole accorte:\\*
— Son lacrime d’amore e non è acqua!
\end{altverse}
\end{poem}

% 68
\begin{poem}{Vurria fare comme fa la quaglia}{}
\settowidth{\versewidth}{Torna il cane al padrone, ma... senza quaglia}
\begin{altverse}
Potessi fare come fa la quaglia:\\*
si posa al suolo e i suoi gusti si piglia.\\*
Il cacciatore ci va a farle la caccia\\*
e dice al cane: — Prendila, forza!\\*
Torna il cane al padrone, ma... senza quaglia:\\*
e il cacciatore collera si piglia!\\*
Non serve far tutto questo sali sali:\\*
quanto più tu t’innalzi, più botte pigli.
\end{altverse}
\end{poem}

% 69
\begin{poem}{Zi’ munacella cu’ ’stu manto pinto}{}
\settowidth{\versewidth}{Zi’ monacella mia, con questo manto pinto}
\begin{altverse}
Zi’ monacella mia, con questo manto pinto\\*
tu dici a tutti che vuoi far la santa!\\*
E poi di anni quanti ne hai? Già venti?\\*
— Voglio il maritiello, tenetevi il manto.\\*
Io voglio farmi una scala di rami leggeri:\\*
lo voglio scalare bene ’sto convento,\\*
lo voglio scalare da fuori e da dentro.\\*
Dov’è zi’ monacella mia, ché non la sento?
\end{altverse}
\end{poem}

\end{document}
